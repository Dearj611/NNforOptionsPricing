\documentclass{report}
\usepackage{setspace}
\usepackage{subfigure}

\pagestyle{plain}
\usepackage{amssymb,graphicx,color}
\usepackage{amsfonts}
\usepackage{latexsym}
\usepackage{a4wide}
\usepackage{amsmath}

\newtheorem{theorem}{THEOREM}
\newtheorem{lemma}[theorem]{LEMMA}
\newtheorem{corollary}[theorem]{COROLLARY}
\newtheorem{proposition}[theorem]{PROPOSITION}
\newtheorem{remark}[theorem]{REMARK}
\newtheorem{definition}[theorem]{DEFINITION}
\newtheorem{fact}[theorem]{FACT}

\newtheorem{problem}[theorem]{PROBLEM}
\newtheorem{exercise}[theorem]{EXERCISE}
\def \set#1{\{#1\} }

\newenvironment{proof}{
PROOF:
\begin{quotation}}{
$\Box$ \end{quotation}}



\newcommand{\nats}{\mbox{\( \mathbb N \)}}
\newcommand{\rat}{\mbox{\(\mathbb Q\)}}
\newcommand{\rats}{\mbox{\(\mathbb Q\)}}
\newcommand{\reals}{\mbox{\(\mathbb R\)}}
\newcommand{\ints}{\mbox{\(\mathbb Z\)}}

%%%%%%%%%%%%%%%%%%%%%%%%%%


\title{{\vspace{-14em} \includegraphics[scale=0.4]{ucl_logo.png}}\\
{{\Huge Machine Learning on Options Pricing}}\\
{\large Optional Subtitle}\\
}
\date{Submission date: 1 April 2020}
\author{Wenwen Zheng\thanks{
{\bf Disclaimer:}
This report is submitted as part requirement for the BSc Computer Science at UCL. It is
substantially the result of my own work except where explicitly indicated in the text.
\emph{Either:} The report may be freely copied and distributed provided the source is explicitly acknowledged
\newline  
\emph{Or:}\newline
The report will be distributed to the internal and external examiners, but thereafter may not be copied or distributed except with permission from the author.}
\\ \\
BSc Computer Science\\ \\
Dr Dariush Hosseini }


\begin{document}
 
\onehalfspacing
\maketitle
\begin{abstract}
This report will provide an overview on fast substitutions to traditional option pricing techniques using DL. In particular, to examine current approaches. Starting with the ‘Deep Learning for Option Pricing’[Robert Culkin & Sanjiv R. Das] and ‘Supervised Deep Neural Networks’[Tugce Karatas, Amir Oskoui, Ali Hirsa], comparisons and assessments can be made in order to generate new ideas from their inspirations and verify my own findings. 

\end{abstract}
\tableofcontents
\setcounter{page}{1}


\chapter{Introduction}
This chapter includes information which covers the big picture of the project, it should include:
\begin{itemize}
\item Motivation; Why this area etc
\item My hypothesis tests
\item Summaries of how I will approach the problem
\item The scope and products of interests
\end{itemize}

\section{The problem}
Options are one important financial instruments for investments. The ability to predict the prices is essential. The industry is still replying on traditional analytic methods to make estimations with assumptions that are known to be wrong. Noticing deep learning's impressing performance on prediction after training, this project thus decides to explore the possibilities of substituting the existing methods by trained neural networks and yield more stable and accurate outputs.

\section{Aims and Deliverables}
This project aims to complete the following objectives:
\begin{itemize}
\item An examination of machine learning approaches to pricing a range of options products will be examined. There will be a particular focus on methods proposed by some recent papers. The focus will be the implementations for vanilla options, then for some exotic extensions.
\item A comparison of pricing speed on novel test points will be made between the ML approaches and traditional numerical approaches (particularly Monte-Carlo based mechanisms).
\item In addition, an examination of how the ML approaches can be used for hedging will be made, and a discussion of advantages and disadvantages vis a vis traditional methods will be considered.
\end{itemize}
After several rounds of experiments and analysis, the aim is to deliver following outputs:
\begin{itemize}
\item A review of literature pertaining to machine learning approaches to options pricing – in particular, NN/DL approaches
\item An implementation of a DL network to price vanilla Black Scholes options and an analysis of Greek stability and accuracy together with pricing speed vs traditional analytic and Monte-Carlo approaches will be made
\item The experiments will be repeated for a portfolio of exotic derivatives
\item Time-permitting, a consideration of implications, for example, for the findings for CVA (credit valuation adjustment) books within banks, will be made
\end{itemize}

\section{Annotated Contents}
This section gives a detailed summary of the project structure and the contents for each chapter.
\textbf{Chapter One: Introduction}
This chapter gives the basic information about the project and provides an overview of this report.
\textbf{Chapter Two: Literature Review}
This chapter details the related background knowledge prior to this project. It covers both financial and computer science related information. Explanations about options and deep learning in general and the works focusing on both aspects are provided. The scope is then limited to the types of options and specific models I would like to use. Some introductions about the traditional methods and existing neural networks for options pricing are mentioned as they will be used as references and comparisons in my assessment and evaluation.
\textbf{Chapter Three: Data and Pre-processing}
This chapter includes everything about the data we use throughout the project. The considerations about using solely synthetic data are elaborated. The chapter walks through the requirements for the datasets and the steps for generating the data. Both vanilla options and exotic options are generated in this project.
\textbf{Chapter Four: Experiments with Different Methods}
Several experiments based on existing papers are carried out in this chapter with new findings and refinements made accordingly. There are trained deep neural networks for both vanilla options and exotic options followed by comparisons to Monte-Carlo, one example of the traditional methods. Other than option pricing, we also look at the Greeks predictions and responding performances. Conclusions from the basic vanilla options are then extended to exotic options.
\textbf{Chapter Five: Evaluations and Conclusions}
Ending the report, the last chapter summaries the project, discusses the success and limitations of the findings, and explore possible future works.

\chapter{Literature Review}

This Chapter contains my research and reading on both options pricing and DL and current DL on options pricing.

\section{Options and Options Pricing}

Contents to be included:
\begin{itemize}
\item what's option (vanilla and exotic etc)
\item BS and the Greeks
\item Options price prediction, the Greeks, Monte-Carlo and its accuracy/stability because of unstable Greeks
\end{itemize}


\subsection{Options, the traditional financial instrument}
Source link: \url {https://www.investopedia.com/options-basics-tutorial-4583012}\newline
An option is a contract giving the buyer the right, but not the obligation, to buy (in the case of a call) or sell (in the case of a put) the underlying asset at a specific price on or before a certain date.
A stock option contract typically represents 100 shares of the underlying stock, but options may be written on any sort of underlying asset from bonds to currencies to commodities.
There are four things you can do with options:
\begin{itemize}
\item Buy calls
\item Sell calls
\item Buy puts
\item Sell puts
\end{itemize}
Buying stock gives you a long position. Buying a call option gives you a potential long position in the underlying stock. Short-selling a stock gives you a short position. Selling a naked or uncovered call gives you a potential short position in the underlying stock.

Buying a put option gives you a potential short position in the underlying stock. Selling a naked, or unmarried, put gives you a potential long position in the underlying stock. Keeping these four scenarios straight is crucial.

People who buy options are called holders and those who sell options are called writers of options. Here is the important distinction between holders and writers:
In this project, we focus on call options and European options only. American options can be exercised at any time between the date of purchase and the expiration date. European options are different from American options in that they can only be exercised at the end of their lives on their expiration date.
We will also look into vanilla options first, then extend the findings to exotic options. 

\section{Deep Learning}
\subsection{Deep Learning in general}
Explain Deep Learning in a simple way\\
Types of Deep Learning (Methods to be used in this report)
\subsection{Deep Learning in Finance & Options Pricing}


\chapter{Data and Pre-processing}
This Chapter contains everything with synthetic data for all experiments;
vanilla options;\\
Monte-Carlo;\\
Exotic options'\\
\section{Validation and Testing}
Validation is needed because of the problem of peeking [\url{https://machinelearningmastery.com/difference-test-validation-datasets/}]. I realise that improvements could be made upon the suggested models in the first paper. 
A good definition of the three data sets was proposed in Ripley’s book “Pattern Recognition and Neural Networks” as follows:
– Training set: A set of examples used for learning, that is to fit the parameters of the classifier.\\
– Validation set: A set of examples used to tune the parameters of a classifier, for example to choose the number of hidden units in a neural network.\\
– Test set: A set of examples used only to assess the performance of a fully-specified classifier.\\
Despite having enough data for validation (as they could be generated endlessly), one seperate, independent validation set has limited ability to identify and assess the uncertainty of a model. [\url{Max Kuhn and Kjell Johnson, Page 78, Applied Predictive Modeling, 2013}]Therefore, cross-validation is preferred.
\begin{definition}\label{def}
See definition~\ref{def}.
\end{definition}
\begin{theorem}
For all $n\in\nats,\; 1^n=1$.
\end{theorem}
\begin{proof}
By induction over $n$.
\end{proof}

\chapter{Experiments with Different Methods}

\section{Deep Learning Model for Option Pricing}
Focus on Deep Learning for Option Pricing’[Robert Culkin & Sanjiv R. Das]\\
Re-implement the DL model on vanilla options and produce the suggested test results\\
Implement traditional Monte-Carlo methods for the same vanilla options \\
Assessments on performances - speed (various data size), accuracy, Greek stability etc
Based on results, make improvements and suggest different user cases\\
Train the DL model and the traditional method for exotic options\\
Run tests and comparisons again for new conclusions

\subsection{DL Implementation on Vanilla options}
\subsection{Validation on the Hyperparameters}
\subsection{Implementation of  traditional Monte-Carlo method}
\subsection{Examination with Traditional Methods}
\subsection{Extension to Exotic Options Pricing}

\section{Supervised Deep Neural Network (DNN)}
Focus on ‘Supervised Deep Neural Networks’[Tugce Karatas, Amir Oskoui, Ali Hirsa]\\
Train the supervised DNN model suggested in the paper and make variations to work for both vanilla options and exotic options\\
Run assessments during several stages and draw conclusions along the process

\subsection{DNN Implementation on Vanilla options}
\subsection{Examination with Traditional Methods}
\subsection{Extension to Exotic Options Pricing}

\section{Possible extensions on CVA}
\subsection{Implementation Details}
\subsection{Result Analysis}


\chapter{Evaluations and Conclusions}
\section{Achievements and Deliverables}
Summarise the achievements to confirm the project goals have been met.
\section{Evaluation}
Evaluation of the work (this may be in a separate chapter if there is substantial evaluation).
\section{Future Work}
How the project might be continued, but don't give the impression you ran out of time!

\appendix


\begin{thebibliography}{HHM99}


\bibitem[Pri70]{PriorNOP70}  %only an example
A.~Prior.
\newblock The notion of the present.
\newblock {\em Studium Generale}, 23:  245--248, 1970.


\bibitem[Rey97]{Rey:D}
M.~Reynolds.
\newblock A decidable temporal logic of parallelism.
\newblock {\em Notre Dame Journal of Formal Logic}, 38(3):  419--436,
  1997.
\end{thebibliography}
\chapter{Other appendices, e.g., code listing}
Put your appendix sections here

\end{document}